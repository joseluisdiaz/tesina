\documentclass[fleqn]{article}

\usepackage[utf8]{inputenc}
\usepackage{cite}

\usepackage{amsfonts}
\usepackage{amsmath}
\usepackage{amssymb}

\usepackage{hyperref}
\usepackage{syntax}
\usepackage{listings}

\lstdefinelanguage{sal}{
  keywords = {send, become, let, new, in, to, if, then, else, case, def, end, case, of}
}

\begin{document}

\subsection{Conceptos básicos}

\subsubsection*{Comportamientos}

\subsubsection*{Creando Actores}

\subsubsection*{Creando comunicaciones}

\subsubsection*{Comandos}

\subsection{Lenguage de Actor Minimo}

Este lenguaje minimo fue mostrado por primera vez en la tesis doctoral
\cite{Agha:1986:AMC:7929}, fue consevido como un lenguaje  con fines pedagógicos.
Un programa en este lenguaje está compuesto por un conjunto de definiciones de
\textit{Comportamientos}. Como la mayoría de los lenguaje de programación,
se agrega un behavior llamado \textbf{main} como punto de entrada.
Estos definen el comportamiento concreto de cada actor en terminos de
expresiones y comandos.

\subsubsection{Expresiones}

Existen tres tipos primitivos, booleanos, enteros y mailbox. Las operaciones
posibles entre los booleanos, \textbf{or}, \textbf{and}, \textbf{not}. Con
respecto a los enteros se pueden operar utilizando \textbf{+}, \textbf{-},
\textbf{*}, \textbf{/}. Un mailbox es un identificador que es devuelto cuando se
crea un nuevo actor.

\subsubsection{Comandos}
La gramática de los comandos en SAL es la siguiente:

\begin{grammar}
<command> ::= `send' $e_1, e_2, ..., e_n$ `to' <actor>
\alt `become' $B(e_1, e_2, ..., e_n)$
\alt `let' $x_1$ = `new' $B_1(e_1, e_2, ..., e_{1n})$, \\ 
 ... $x_k$ = `new' $B_k(e_1, e_2, ..., e_{kn})$        \\
 `in' <command> 
\alt `if` <bool-expr> `then' <command> `else' <command> `end if' 
\alt <command> `;' <command>
\end{grammar}

\begin{description}
\item [send]  Este comando permite enviar mensajes a otros actores, toma como
  parametro una lista separada por coma de las expresiones a enviar, y el actor
  destino, el envio de mensajes es asincronico. Cada expresion es evaluada antes
  de ser enviada.
\item [become] Este comando especifica el siguiente comportamiendo del actor
  que está procesando el mensaje recicibido. Como en el caso anterior se evaluan
  las expresiones antes de ser enviadas, y estas apareceran como la listas de
  parametros del comportamiento. 
\item[new] Este comando sirve para crear nuevos. El alcance de los
  identificadores de los nuevos actores creados está sujeto a el cuerpo a
  el cuerpo del \textbf{let}.
\item[condicional] La expresion booleana tiene que ser evaluada, si es
\item[secuenciación] a diferencia de la semantica original de SAL, los
  comandos separados por punto y coma ocurren secuancialmente. \footnote{hoy
    tengo escrito que las cosas ocurran secuencialmente dentro del actor, ya que
  lo reemplazao usando ->, podría usar el operador de interleaving the CSP para
  guardar la semantica concreata originalmente propuesta por Agah.}
\end{description}
  verdadera tiene que ejecutarse el primer comando, en caso contrario el otro.
-- Agregar la idea de que los comandos ocurren en cualquier orden? ---

\subsubsection{Comportamientos}

La sintaxis de los comportamientos es la siguiente:

\begin{grammar}
  <behavior> :== `def' <beh name> `(' <param-list> `)' `[`'<communication-list>`]' \\
            <command>* \\
  `end def'
\end{grammar}

La lista de parametros \textit{param-list} es una lista de variables separadas por coma
que se inicializan cuando el actor es creado. La lista de comunicaciones (communication-list)
muchas veces depende del tipo de comunicación que se esté enviando, por ejemplo
si se está enviando un mensaje de \textbf{Retiro} a un actor simulando ser una
caja de ahorros la comunicación debe especificar la cantidad a retirar. Pero si
quiere saber cual es el \textbf{Balance} no debe especificar ningun parametro
adicional.
Basicamente se bifurca por el valor de uno de los campos llamado
\textbf{tag-field}, dependiendo de este diferentes identificarores son esperados.
La sintaxis de esta lista es la siguiente:

\begin{grammar}
  <params> ::= <id> | <id> `,' <params>
  <var-list> ::= `case' <tag-field> `of' <variant>+ `end case'
  <variant> ::= <case label> `:' <params>
\end{grammar}

El siguiente ejemplo muestra un caso donde es útil utilizar la sintaxis basada
en \textbf{case}:

\begin{verbatim}
case request of 
  depositar : (cliente, monto) 
  retirar: (cliente, monto) 
  balance: (cliente) 
end case
\end{verbatim}

En este caso \textbf{depositar}, \textbf{retirar} y \textbf{balance} son
simbolos. 
No siempre es necesario utilizar esta sintaxis, communication-list puede tener
la misma estructura que param-list.

\subsubsection{Ejemplo}

SAL no tiene construcciones para control de flujo, pero esto se puede encodear
dentro como mensajes.

--- INCLUIR UNA IMAGEN DE COMO SE CALCULA EL FACTORIAL DE 3 ---

La implementación dada a continuación esta tomada de \cite{Agha:1986:AMC:7929}.


\begin{lstlisting}[language=sal]
def Factorial()[val, customer]
  if val = 0 then
    send [1] to customer
  else
    let cont = new FactorialCont(val, customer)
       in send [val - 1, cont] to self
  end if
  become Factorial()
end def

def FactorialCont(n, customer)[arg] 
  send [n * arg] to customer
end def
\end{lstlisting}

Cuando \textbf{Factorial} recibe un mensaje esté incluye un entero positivo y
una referencia al actor al que el resultado debe ser enviado. Al recibir el
mensaje se comporta de la siguiente manera, si el valor que recibe es 0, le
envia a customer 1. En caso contrario, crea un nuevo actor
\textbf{FactorialCont} con los parametros, \textbf{val - 1} y \textbf{customer}.
Cuando eventualmente \textbf{FactorialCont} recibe un entero, le envia a
\textbf{Customer} la multiplicación de este por el valor que tenía originalmente
en parametro. Se puede ver en la figura como esto evoluciona para el calculo del
factorial de 3.

\subsection{CSP y actores}
Para empezar a describir como se modeló en CSP, primero es importante hacer
referencia a que no se utilizó CSPm, que combina los operadores de CSP
originalmente propuesto por Hoare, y un lenguage funcional.

runtime: 
\begin{enumerate}
\item  introducir la idea de mailbox y los canales de comunicación
\item introducir el tipo ActorId y explicar la limitción sobre la red de actores
(tenemos que saber cuantos actores necesitamos)
\item introducir la separación entre el pedido de creación y el comienzo del actor.
\item funcionalidad de main
\end{enumerate}

\subsection{Generalizando el ejemplo}

---- Definir \textbf{translateExp}  ----

La clase \textbf{Cmnd} con elementos de tipo S está dada por:

\begin{verbatim}
S :== S_1 ; S_2 | if b then S_1 else S_2 | send [e1, .., e_i] to a | become new
E(e_1, .. ,e_i) | let a_1 = new E_1(e_1,..,e_i) and ... a_j = new
E_1(e_1,..,e_i) { S } 
\end{verbatim}

definimos la funcion \textbf{translateCmd} de la siguiente forma:

\begin{verbatim}
translateCmd (S_1 S_2) = translateCmd(S_1) -> translateCmd(S_2)
\end{verbatim}


\begin{verbatim}
translateCmd(if b then S_1 else S_2) = 
   if (translateExp(b)) then
       translateCmd(S_1) else 
       translateCmd(S_2)
\end{verbatim}

\begin{verbatim}
translateCmd(send[e_1, ..., e_i] to a) = CommSend.a.
         (translateExp(e_1), ..., 
          translateExp(e_i)) 
\end{verbatim}

\begin{verbatim}
translateCmd(become new Beh(e_1, ..., e_n)) = runningBeh(self, e_1, ..., e_n)
\end{verbatim}

newEnv es el resultado de agregar a el entorno de las variables de mailbox $a_1
= E_1.pid_1$ .. $a_n = E_n.pid_n$
\begin{verbatim}
translateCmd(let a_1 = new E_1(e_1, ..., e_j) and 
         ... and a_n = new E_N(e_1, ..., e_j) { S } = 

CreateAsk?E_1.pid_1!(translateExp(e_1), ...,translateExp(e_j)) ->
CreateAsk?E_N.pid_n!(translateExp(e_1), ...,translateExp(e_j)) ->
translateCmd(S, newEnv)
\end{verbatim}


La clase \textbf{Beha} con elementos de tipo S está dada por:

\begin{verbatim}
def behName(a_1 .. a_i)[n_1 ... n_j]
  S
end def
\end{verbatim}

Tendria como equivalente en CSP:

\begin{verbatim}

behName = ||| actorId : {|BehName|} @ Create.actorId?(a_1, ..., a_i) ->
behNameRunning(actorId, a_1, .., a_n)

behNameRunning(self, a_1, .., a_n) = CommRecv.self(n_1 ... n_j) -> translateCmd(S)

\end{verbatim}

--- Agregar case? ---


\bibliography{references}{}
\bibliographystyle{plain}
\end{document}

