\documentclass[fleqn]{article}

\usepackage[utf8]{inputenc}
\usepackage{cite}

\usepackage{amsfonts}
\usepackage{amsmath}
\usepackage{amssymb}

\usepackage{hyperref}


\newcommand{\myList}[1]{\langle #1 \rangle}          
\newcommand{\myCons}[0]{\mathbin{\raise 0.8ex\hbox{$\smallfrown$}}}
\newcommand{\paralleln}[2]{\overset{#2}{\underset{i = #1}{\parallel}}}
\newcommand{\boxn}[2]{\overset{#2}{\underset{i = #1}{\Box}}}


\begin{document}

Los actores son naturalmente asincronos, la recepcion del mensaje y cuando se consume ocurren en momentos diferentes. Q es quien recibe el mensaje que luego será consumido por el actor en cuestion. Ningún proceso de CSP envia directamente un mensaje a otro actor, todos los mensajes se envian utilizando Q. Cada proceso $Q_i$ cumple la funcion de ``mailbox'' para un actor especficio, toda comunicacion pasa por el. \\

\begin{gather*}
Mailbox_i(\myList{}) = external.i?msg \rightarrow Mailbox_i(\myList{msg}) \\
Mailbox_i(\myList{h} \myCons TAIL) = external.i?msg \rightarrow Mailbox_i(\myList{msg} \myCons \myList{h} \myCons TAIL) \\ 
\qquad \square \; internal.i!h \rightarrow Mailbox_i(TAIL) \\  
MAILBOXS = \paralleln{1}{m} actors?start_q(i) \rightarrow Mailbox_i(\myList{})
\end{gather*}

Este modelo cuenta con dos canales, $internal$ el cual está destinado a la comunicación entre la cola y el actor apareado a la misma y $external$ que es canal por el cual se reciben los mensajes que envian otros actores. Notar que la sincronización se hace en términos de ``i'' que es el mailbox con el cual es identificado este proceso.

\begin{gather*}
recCustomer_i(n, client) = internal.i?(k) \rightarrow external.client!(k*n) \rightarrow STOP \\
recCustomer = \paralleln{1}{m} actors?start_{rc}(i, n, client) \rightarrow recCustomer_i(n, client)
\end{gather*}

$recCustomer_i$ recibe un mensaje por su canal interno. Su única funcion es enviar un mensaje a $client$ con el valor del entero que recibio, $k$ multiplicado por el parametro $n$ el cual oportunamente recibio cuando fue creado. Esta multiplicación claramente engorrosa, intenta describir el espiritu del paralelismo un actor que es una suerte de proceso efectua una computación y envia el resultado, si bien por como se envian los mensajes todo es secuencial es una aproximación a la idea de paralelismo.

\begin{gather*} 
factorialRec_i = internal.i?(k,client) \rightarrow \\
\qquad factorialRecCaso0_i(client) \\
\qquad \qquad \lceil k == 0 \rceil \\
\qquad factorialRecCasoN_i(client, k, createNewMailbox())
\end{gather*}

$factorialRec_i$ recibe una tupla con dos valores, un entero $k$ y un mailbox $client$. Si este valor fuera cero, se comporta como $factorialRecCaso0$ sino como $factorialRecCasoN$.
La función externa $createNewMailbox$ devuelve un numero de mailbox no usado. 

\begin{gather*}
factorialRecCaso0_i(client) = external.client!1 \rightarrow factorialRec_i 
\end{gather*}

Cuando se comporta como $factorialRecCaso0_i$ envía al mailbox que recibio como parametro el valor $1$ y luego se vuelve a comportar como $factorialRec_i$. Sería el fin de la recursión, para el valor $0$ enviar al cliente $1$.

\begin{gather*}
factorialRecCasoN_i(client,k, newMailbox) = \\
  \qquad actors!start_q(newMailbox) \rightarrow actors!start_{rc}(newMailbox, k, client) \rightarrow \\
  \qquad external.i!(k - 1, newMailBox) \rightarrow factorialRec_i
\end{gather*}

En el caso $factorialRecCasoN$, este se llama con tres parametros, mailbox del cliente, el valor entero que recibio y un nuevo mailbox creado en el paso anterior. Este paso ``instancia'' un actor y una cola utilizando los mensajes espaciales $start_q$ y $start_{rc}$. Una vez enviado estos mensajes, se auto-envia un mensaje que contiene $k - 1$ y el valor del nuevo mailbox creado.\\

\begin{gather*}
factorialRecStart(i) = actors!start_q(i) \rightarrow factorialRec_i \\
\end{gather*}

En este caso $factorialRecStart$ es el actor representando el factorial, con el mailbox $i$, un parametro que recibirá en el momento que se despierte.

\begin{gather*}
client(mailboxClient, mailboxFactorial) =  actors!start_q(mailboxClient) \rightarrow  \\
\qquad  external.mailboxFactorial!(mailboxClient,NUM) \rightarrow \\
\qquad internal.mailboxClient?k \rightarrow HACERALGOCON(k) 
\end{gather*}

El actor que utiliza el factorial, aqui denominado $client$. Una vez inicializado le envia utilizando el canal $external$ al mailbox de factorial, la tupla comprendida con su propio mailbox y el valor del cual quisieramos calcular el factorial, en este caso representado por $NUM$ debería ser reemplazado por cualquier natural. 

\begin{gather*}
S = START(createNewMailbox, createNewMailbox) \\
START(mailboxClient, mailboxFactorial) = factorialRecStart(mailboxFactorial) \parallel \\
\qquad client(mailboxClient, mailboxFactorial) 
\end{gather*}

Por cuestiones de inicialización, necesitamos conocer en el momento de la creacion del actor $client$ el mailbox de $factorialRec$. Estas dos formulas capturan ese comportamiento.

\begin{gather*}
SISTEMA = MAILBOX \parallel recCustomer \parallel factorialRec \parallel S
\end{gather*}

SISTEMA representa todo el conjunto de actores y mailbox. Veamos un ejemplo donde se calcula el factorial de 0.

\begin{gather*}
client(mailboxClient, mailboxFactorial) =  actors!start_q(mailboxClient) \rightarrow  \\
\qquad external.mailboxFactorial!(mailboxClient,0) \rightarrow \\
\qquad internal.mailboxClient?k \rightarrow HACERALGOCON(k) 
\end{gather*}

La anterior formula surge de cambiar en $client$, $NUM$ por el valor 0. Este sería nuestro nuevo cliente. NOTA: Como en este caso no serían necesarios $recCustomer$ 

\begin{gather*}
MAILBOX \parallel factorialRec \parallel S
\end{gather*}

$\qquad \qquad \equiv$

\begin{gather*}
actors?start_q(i) \rightarrow Mailbox_i(\myList{})  \parallel \\
START(createNewMailbox, createNewMailbox) 
\end{gather*}

$\qquad \qquad \equiv$

\begin{gather*}
( \\ 
\quad actors?start_q(1) \rightarrow Mailbox_1(\myList{}) \parallel \\
\quad actors?start_q(2) \rightarrow Mailbox_2(\myList{}) .. )  \\
) \parallel \\
( \\
\quad factorialRecStart(mailboxFactorial) \parallel \\
\quad client(mailboxClient, mailboxFactorial) \\ 
)
\end{gather*}

$\qquad \qquad \equiv$

\begin{gather*}
( \\ 
\quad actors?start_q(1) \rightarrow Mailbox_1(\myList{}) \parallel \\
\quad actors?start_q(2) \rightarrow Mailbox_2(\myList{}) .. )  \\
) \parallel \\
\quad ( actors!start_q(1) \rightarrow factorialRec_1 ) \parallel \\
( \\
\quad actors!start_q(2) \rightarrow  \\
\quad external.1!(2,0) \rightarrow \\
\quad internal.2?k \rightarrow HACERALGOCON(k) \\ 
)
\end{gather*}


$\qquad \qquad \equiv$

\begin{gather*}
( \\ 
\quad external.1?msg \rightarrow Mailbox_1(\myList{msg})  \parallel \\
\quad actors?start_q(2) \rightarrow Mailbox_2(\myList{}) )  \\
) \parallel \\
\quad ( actors!start_q(1) \rightarrow factorialRec_1 ) \parallel \\
( \\
\quad external.1!(2,0) \rightarrow \\
\quad internal.2?k \rightarrow HACERALGOCON(k) \\ 
)
\end{gather*}

$\qquad \qquad \equiv$

\begin{gather*}
( \\ 
\quad external.1?msg \rightarrow Mailbox_1(\myList{msg})  \parallel \\
\quad external.2?msg \rightarrow Mailbox_2(\myList{msg})  \\
) \parallel \\
(
\quad internal.i?(k,client) \rightarrow \\
\quad factorialRecCaso0_i(client) \\
\quad \qquad \lceil k == 0 \rceil \\
\quad factorialRecCasoN_i(client, k, createNewMailbox()) \\
) \parallel \\
( \\
\quad external.1!(2,0) \rightarrow \\
\quad internal.2?k \rightarrow HACERALGOCON(k) \\ 
)
\end{gather*}

$\qquad \qquad \equiv$

\begin{gather*}
( \\ 
external.1?msg \rightarrow Mailbox_1(\myList{msg} \myCons \myList{(2,0)} \myCons TAIL) \\ 
\quad \square \; internal.1!(2,0) \rightarrow Mailbox_1(TAIL) \\  
) \parallel \\
\quad external.2?msg \rightarrow Mailbox_2(\myList{msg})  \\
) \parallel \\
(
\quad internal.i?(k,client) \rightarrow \\
\quad factorialRecCaso0_1(client) \\
\quad \qquad \lceil k == 0 \rceil \\
\quad factorialRecCasoN_1(client, k, createNewMailbox()) \\
) \parallel \\
( \\
\quad internal.2?k \rightarrow HACERALGOCON(k) \\ 
)
\end{gather*}

$\qquad \qquad \equiv sincronizando en internal.1!(2,0)$

\begin{gather*}
( \\ 
\quad external.1?msg \rightarrow Mailbox_1(\myList{msg}) \parallel \\
\quad external.2?msg \rightarrow Mailbox_2(\myList{msg})  \\
) \parallel \\
(
\quad factorialRecCaso0_1(2) \\
\quad \qquad \lceil 0 == 0 \rceil \\
\quad factorialRecCasoN_1(1, 0, createNewMailbox()) \\
) \parallel \\
( \\
\quad internal.2?k \rightarrow HACERALGOCON(k) \\ 
)
\end{gather*}

$\qquad \qquad \equiv$

\begin{gather*}
( \\ 
\quad external.1?msg \rightarrow Mailbox_1(\myList{msg}) \parallel \\
\quad external.2?msg \rightarrow Mailbox_2(\myList{msg})  \\
) \parallel \\
( \\
\quad external.2!1 \rightarrow factorialRec_1
) \parallel \\
( \\
\quad internal.2?k \rightarrow HACERALGOCON(k) \\ 
)
\end{gather*}

$\qquad \qquad \equiv sincronizando external.2?1$

\begin{gather*}
( \\ 
\quad external.1?msg \rightarrow Mailbox_1(\myList{msg}) \parallel \\
( \\
external.2?msg \rightarrow Mailbox_2(\myList{msg} \myCons \myList{1} \myCons TAIL) \\ 
\quad \square \; internal.2!1 \rightarrow Mailbox_2(TAIL) ) \\
) \parallel \\
( \\
\quad internal.i?(k,client) \rightarrow \\
\quad factorialRecCaso0_1(client) \\
\quad \qquad \lceil k == 0 \rceil \\
\quad factorialRecCasoN_1(client, k, createNewMailbox()) \\
) \parallel \\
( \\
\quad internal.2?k \rightarrow HACERALGOCON(k) \\ 
)
\end{gather*}

$\qquad \qquad \equiv sincronizando internal.2?1$

\begin{gather*}
( \\ 
\quad external.1?msg \rightarrow Mailbox_1(\myList{msg}) \parallel \\
\quad external.2?msg \rightarrow Mailbox_2(\myList{msg})  \\
) \parallel \\
( \\
\quad internal.i?(k,client) \rightarrow \\
\quad factorialRecCaso0_1(client) \\
\quad \qquad \lceil k == 0 \rceil \\
\quad factorialRecCasoN_1(client, k, createNewMailbox()) \\
) \parallel \\
\quad  \rightarrow HACERALGOCON(1) \\ 
)
\end{gather*}



\bibliography{references}{}
\bibliographystyle{plain}
\end{document}
