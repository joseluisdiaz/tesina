\documentclass[fleqn]{article}

\usepackage[utf8]{inputenc}
\usepackage{cite}

\usepackage{amsfonts}
\usepackage{amsmath}
\usepackage{amssymb}

\usepackage{hyperref}

\newcommand{\myList}[1]{\langle #1 \rangle}          
\newcommand{\myCons}[0]{\mathbin{\raise 0.8ex\hbox{$\smallfrown$}}}
\newcommand{\paralleln}[2]{\overset{#2}{\underset{i = #1}{\parallel}}}
\newcommand{\boxn}[2]{\overset{#2}{\underset{i = #1}{\Box}}}


\begin{document}

Los actores son naturalmente asincronos, la recepcion del mensaje y cuando se consume ocurren en momentos diferentes. Q es quien recibe el mensaje que luego será consumido por el actor en cuestion. Ningún proceso de CSP envia directamente un mensaje a otro actor, todos los mensajes se envian utilizando Q. Cada proceso $Q_i$ cumple la funcion de ``mailbox'' para un actor especficio, toda comunicacion pasa por el. \\

\begin{gather*}
Q_i(\myList{}) = external.i?msg \rightarrow Q_i(\myList{msg}) \\
Q_i(\myList{h} \myCons TAIL) = external.i?msg \rightarrow Q_i(\myList{msg} \myCons \myList{h} \myCons TAIL) \\ 
\qquad \square \; internal.i?!h \rightarrow Q_i(TAIL) \\  
Queue = \paralleln{1}{m} actors?start_q(i) \rightarrow Q_i(\myList{})
\end{gather*}

Este modelo cuenta con dos canales, $internal$ el cual está destinado a la comunicación entre la cola y el actor apareado a la misma y $external$ que es canal por el cual se reciben los mensajes que envian otros actores. Notar que la sincronización se hace en términos de ``i'' que es el mailbox con el cual es identificado este proceso.

\begin{gather*}
recCustomer_i(n, client) = internal.i?(k) \rightarrow external.client!(k*n) \rightarrow STOP \\
recCustomer = \paralleln{1}{m} actors?start_{rc}(i, n, client) \rightarrow recCustomer_i(n, client)
\end{gather*}

$recCustomer_i$ recibe un mensaje por su canal interno. Su única funcion es enviar un mensaje a $client$ con el valor del entero que recibio, $k$ multiplicado por el parametro $n$ el cual oportunamente recibio cuando fue creado. Esta multiplicación claramente engorrosa, intenta describir el espiritu del paralelismo un actor que es una suerte de proceso efectua una computación y envia el resultado, si bien por como se envian los mensajes todo es secuencial es una aproximación a la idea de paralelismo.

\begin{gather*}
factorialRec_i = internal.i?(k,client) \rightarrow \\
\qquad factorialRecCaso0_i(client) \\
\qquad \qquad \lceil k == 0 \rceil \\
\qquad factorialRecCasoN_i(client, k, createNewMailbox(i))
\end{gather*}

$factorialRec_i$ recibe una tupla con dos valores, un entero $k$ y un mailbox $client$. Si este valor fuera cero, se comporta como $factorialRecCaso0$ sino como $factorialRecCasoN$. 

\begin{gather*}
factorialRecCaso0_i(client) = external.client!1 \rightarrow factorialRec_i 
\end{gather*}

Cuando se comporta como $factorialRecCaso0_i$ envía al mailbox que recibio como parametro el valor $1$ y luego se vuelve a comportar como $factorialRec_i$. Sería el fin de la recursión, para el valor $0$ enviar al cliente $1$.

\begin{gather*}
factorialRecCasoN_i(client,k, newMailbox) = \\
  \qquad actors!start_q(newMailbox) \rightarrow actors!start_{rc}(newMailbox, k, client) \rightarrow \\
  \qquad external.i!(k - 1, newMailBox) \rightarrow factorialRec_i
\end{gather*}

En el caso $factorialRecCasoN$, este se llama con tres parametros, mailbox del cliente, el valor entero que recibio y un nuevo mailbox creado en el paso anterior. Este paso ``instancia'' un actor y una cola utilizando los mensajes espaciales $start_q$ y $start_{rc}$. Una vez enviado estos mensajes, se auto-envia un mensaje que contiene $k - 1$ y el valor del nuevo mailbox creado.\\

\begin{gather*}
factorialRec =
  actors!start_q(1) \rightarrow factorialRec_1 \\
CLIENT =
  actors!start_q(2) \rightarrow external.1!(2,5) \rightarrow \\
\qquad internal.2?k \rightarrow HACERALGOCON(k) \\
SISTEMA = Queue \parallel recCustomer \parallel FACT \parallel CLIENT
\end{gather*}

En este caso $factorialRec$ es el actor representando el factorial, con el mailbox $1$. $CLIENT$ es, quien tiene el mailbox $2$, consulta por el mailbox de $5$ y se queda esperando por el canal interno la respuesta.

\bibliography{references}{}
\bibliographystyle{plain}
\end{document}
