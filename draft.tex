\documentclass[fleqn]{article}

\usepackage[utf8]{inputenc}
\usepackage{cite}

\usepackage{amsfonts}
\usepackage{amsmath}
\usepackage{amssymb}

\usepackage{hyperref}


\newcommand{\myList}[1]{\langle #1 \rangle}          
\newcommand{\myCons}[0]{\mathbin{\raise 0.8ex\hbox{$\smallfrown$}}}
\newcommand{\paralleln}[2]{\overset{#2}{\underset{i = #1}{\parallel}}}
\newcommand{\boxn}[2]{\overset{#2}{\underset{i = #1}{\Box}}}


\begin{document}

Los actores son naturalmente asincronos, la recepcion del mensaje y cuando se consume ocurren en momentos diferentes. Q es quien recibe el mensaje que luego será consumido por el actor en cuestion. Ningún proceso de CSP envia directamente un mensaje a otro actor, todos los mensajes se envian utilizando Q. Cada proceso $Q_i$ cumple la funcion de ``mailbox'' para un actor especficio, toda comunicacion pasa por el. \\

\begin{gather*}
Mailbox_i(\myList{}) = external.i?msg \rightarrow Mailbox_i(\myList{msg}) \\
Mailbox_i(\myList{h} \myCons TAIL) = external.i?msg \rightarrow Mailbox_i(\myList{msg} \myCons \myList{h} \myCons TAIL) \\ 
\qquad \square \; internal.i!h \rightarrow Mailbox_i(TAIL) \\  
MAILBOXS = \paralleln{1}{m} actors?start_q(i) \rightarrow Mailbox_i(\myList{})
\end{gather*}

Este modelo cuenta con dos canales, $internal$ el cual está destinado a la comunicación entre la cola y el actor apareado a la misma y $external$ que es canal por el cual se reciben los mensajes que envian otros actores. Notar que la sincronización se hace en términos de ``i'' que es el mailbox con el cual es identificado este proceso.

\begin{gather*}
recCustomer_i(n, client) = internal.i?(k) \rightarrow external.client!(k*n) \rightarrow STOP \\
recCustomer = \paralleln{1}{m} actors?start_{rc}(i, n, client) \rightarrow recCustomer_i(n, client)
\end{gather*}

$recCustomer_i$ recibe un mensaje por su canal interno. Su única funcion es enviar un mensaje a $client$ con el valor del entero que recibio, $k$ multiplicado por el parametro $n$ el cual oportunamente recibio cuando fue creado. Esta multiplicación claramente engorrosa, intenta describir el espiritu del paralelismo un actor que es una suerte de proceso efectua una computación y envia el resultado, si bien por como se envian los mensajes todo es secuencial es una aproximación a la idea de paralelismo.

\begin{gather*} 
factorialRec_i = internal.i?(k,client) \rightarrow \\
\qquad factorialRecCaso0_i(client) \\
\qquad \qquad \lceil k == 0 \rceil \\
\qquad factorialRecCasoN_i(client, k, createNewMailbox())
\end{gather*}

$factorialRec_i$ recibe una tupla con dos valores, un entero $k$ y un mailbox $client$. Si este valor fuera cero, se comporta como $factorialRecCaso0$ sino como $factorialRecCasoN$.
La función externa $createNewMailbox$ devuelve un numero de mailbox no usado. 

\begin{gather*}
factorialRecCaso0_i(client) = external.client!1 \rightarrow factorialRec_i 
\end{gather*}

Cuando se comporta como $factorialRecCaso0_i$ envía al mailbox que recibio como parametro el valor $1$ y luego se vuelve a comportar como $factorialRec_i$. Sería el fin de la recursión, para el valor $0$ enviar al cliente $1$.

\begin{gather*}
factorialRecCasoN_i(client,k, newMailbox) = \\
  \qquad actors!start_q(newMailbox) \rightarrow actors!start_{rc}(newMailbox, k, client) \rightarrow \\
  \qquad external.i!(k - 1, newMailBox) \rightarrow factorialRec_i
\end{gather*}

En el caso $factorialRecCasoN$, este se llama con tres parametros, mailbox del cliente, el valor entero que recibio y un nuevo mailbox creado en el paso anterior. Este paso ``instancia'' un actor y una cola utilizando los mensajes espaciales $start_q$ y $start_{rc}$. Una vez enviado estos mensajes, se auto-envia un mensaje que contiene $k - 1$ y el valor del nuevo mailbox creado.\\

\begin{gather*}
factorialRecStart(i) = actors!start_q(i) \rightarrow factorialRec_i \\
\end{gather*}

En este caso $factorialRecStart$ es el actor representando el factorial, con el mailbox $i$, un parametro que recibirá en el momento que se despierte.

\begin{gather*}
client(mailboxClient, mailboxFactorial, NUM) =  actors!start_q(mailboxClient) \rightarrow  \\
\qquad  external.mailboxFactorial!(mailboxClient,NUM) \rightarrow \\
\qquad internal.mailboxClient?k \rightarrow HACERALGOCON(k) 
\end{gather*}

El actor que utiliza el factorial, aqui denominado $client$. Una vez inicializado le envia utilizando el canal $external$ al mailbox de factorial, la tupla comprendida con su propio mailbox y el valor del cual quisieramos calcular el factorial, en este caso representado por $NUM$ debería ser reemplazado por cualquier natural. 

\begin{gather*}
  START(mailboxClient, mailboxFactorial, NUM) = \\
  \qquad factorialRecStart(mailboxFactorial) \parallel \\
  \qquad client(mailboxClient, mailboxFactorial, NUM) 
\end{gather*}

Por cuestiones de inicialización, necesitamos conocer en el momento de la creacion del actor $client$ el mailbox de $factorialRec$. Estas dos formulas capturan ese comportamiento.

\begin{gather*}
  SISTEMA(NUM) = MAILBOX \parallel
  recCustomer \parallel
  factorialRec \parallel \\
  START(createNewMailbox(), createNewMailbox(), NUM) \label{start}
\end{gather*}

\newcommand{\mbc}[0]{\mathsf{client}}
\newcommand{\mbf}[0]{\mathsf{factorial}}

SISTEMA representa todo el conjunto de actores y mailboxes. Veamos un ejemplo
donde se calcula el factorial de 0. Cuando se procesa \ref{start} en ambas llamandas a la función
externa $createNewMailbox$ devuelve $0$ en la primer
llamada y $1$ en la segunda, a los efectos de amenizar la lectura se utiliza
$\mbc$ para representar el valor 0 y $\mbf$ para representar a 1.\\

Expandiendo $SISTEMA(0)$ nos queda la siguiente formula:
\begin{gather*}
MAILBOX \parallel
  recCustomer \parallel
  factorialRec \parallel \\
  START(\mbc, \mbf, 0)
\end{gather*}

Expandiendo definiciones de $MAILBOX$, $factorialRec$ y $START$

\begin{gather*}
actors?start_q(\mbc) \rightarrow Mailbox_\mbc(\myList{}) \parallel \\
actors?start_q(\mbf) \rightarrow Mailbox_\mbf(\myList{}) \parallel \\
START(\mbc, \mbf, 0) 
\end{gather*}

Expandiendo expandiendo $START(\mbc,\mbc, 0)$, recordemos que el 0 es el
factorial a calcular.

\begin{gather*}
actors?start_q(\mbc) \rightarrow Mailbox_\mbc(\myList{}) \parallel \\
actors?start_q(\mbf) \rightarrow Mailbox_\mbf(\myList{}) \parallel \\
actorialRecStart(\mbf) \parallel \\
client(\mbc, \mbf, 0)
\end{gather*}

Expandiendo $factorialRecStart$ y $client$.

\begin{gather*}
actors?start_q(\mbc) \rightarrow Mailbox_\mbc(\myList{}) \parallel \\
actors?start_q(\mbf) \rightarrow Mailbox_\mbf(\myList{}) \parallel \\ \\
actors!start_q(\mbf) \rightarrow factorialRec_\mbf \parallel \\ \\
actors!start_q(\mbc) \rightarrow external.\mbf!(\mbc,0) \rightarrow \\ 
\quad internal.\mbc?k \rightarrow HACERALGOCON(k)
\end{gather*}

Sincronizando en $actors?start_q(\mbc)$ 

\begin{gather*}
external.\mbc?msg \rightarrow Mailbox_\mbc(\myList{msg})  \parallel \\
actors?start_q(\mbf) \rightarrow Mailbox_\mbf(\myList{}) \parallel \\
actors!start_q(\mbf) \rightarrow factorialRec_\mbf ) \parallel \\
external.\mbf!(\mbc,0) \rightarrow \\
internal.\mbc?k \rightarrow HACERALGOCON(k) 
\end{gather*}

Sincronizando en $actors?start_q(\mbf)$ y exapndiendo $factorialRec_\mbf$

\begin{gather*}
( \\ 
\quad external.\mbc?msg \rightarrow Mailbox_\mbc(\myList{msg})  \parallel \\
\quad external.\mbf?msg \rightarrow Mailbox_\mbf(\myList{msg})  \\
) \parallel \\
(
\quad internal.\mbf?(k,client) \rightarrow \\
\quad factorialRecCaso0_\mbf(client) \\
\quad \qquad \lceil k == 0 \rceil \\
\quad factorialRecCasoN_\mbf(client, k, createNewMailbox()) \\
) \parallel \\
( \\
\quad external.\mbf!(\mbc,0) \rightarrow \\
\quad internal.\mbc?k \rightarrow HACERALGOCON(k) \\ 
)
\end{gather*}

Sincronizando en $external.\mbf!(\mbc, 0)$

\begin{gather*}
( \\ 
\quad external.\mbc?msg \rightarrow Mailbox_\mbc(\myList{msg}) \parallel  \\
\quad external.\mbf?msg \rightarrow Mailbox_\mbf(\myList{msg} \myCons \myList{(\mbc,0)} \myCons TAIL) \\ 
\quad \square \; internal.\mbf!(\mbc,0) \rightarrow Mailbox_\mbf(TAIL) \\  
) \\
) \parallel \\
(
\quad internal.\mbf?(k,client) \rightarrow \\
\quad factorialRecCaso0_\mbf(client) \\
\quad \qquad \lceil k == 0 \rceil \\
\quad factorialRecCasoN_\mbf(client, k, createNewMailbox()) \\
) \parallel \\
( \\
\quad internal.\mbc?k \rightarrow HACERALGOCON(k) \\ 
)
\end{gather*}

Sincronizando en $internal.\mbf!(\mbc,0)$

\begin{gather*}
( \\ 
\quad external.\mbc?msg \rightarrow Mailbox_\mbc(\myList{msg}) \parallel \\
\quad external.\mbf?msg \rightarrow Mailbox_\mbf(\myList{msg})  \\
) \parallel \\
(
\quad factorialRecCaso0_\mbf(\mbc) \\
\quad \qquad \lceil 0 == 0 \rceil \\
\quad factorialRecCasoN_\mbf(\mbc, 0, createNewMailbox()) \\
) \parallel \\
( \\
\quad internal.\mbc?k \rightarrow HACERALGOCON(k) \\ 
)
\end{gather*}

Expandiendo $factorialRecCaso0_\mbf(\mbc)$

\begin{gather*}
( \\ 
\quad external.\mbc?msg \rightarrow Mailbox_\mbc(\myList{msg}) \parallel \\
\quad external.\mbf?msg \rightarrow Mailbox_\mbf(\myList{msg})  \\
) \parallel \\
( \\
\quad external.\mbc!\mbc \rightarrow factorialRec_\mbf
) \parallel \\
( \\
\quad internal.\mbc?k \rightarrow HACERALGOCON(k) \\ 
)
\end{gather*}

Sincronizando $external.\mbc?\mbc$

\begin{gather*}
( \\ 
external.\mbc?msg \rightarrow Mailbox_\mbc(\myList{msg} \myCons \myList{\mbc} \myCons TAIL) \\ 
\quad \square \; internal.\mbc!\mbc \rightarrow Mailbox_\mbc(TAIL)  \parallel \\
\quad external.\mbf?msg \rightarrow Mailbox_\mbf(\myList{msg}) \parallel \\
) \parallel \\
( \\
\quad internal.\mbf?(k,client) \rightarrow \\
\quad factorialRecCaso0_\mbf(client) \\
\quad \qquad \lceil k == 0 \rceil \\
\quad factorialRecCasoN_\mbf(client, k, createNewMailbox()) \\
) \parallel \\
( \\
\quad internal.\mbc?k \rightarrow HACERALGOCON(k) \\ 
)
\end{gather*}

Sincronizando en $internal.\mbc?\mbc$

\begin{gather*}
( \\ 
\quad external.\mbc?msg \rightarrow Mailbox_\mbc(\myList{msg}) \parallel \\
\quad external.\mbf?msg \rightarrow Mailbox_\mbf(\myList{msg})  \\
) \parallel \\
( \\
\quad internal.\mbf?(k,client) \rightarrow \\
\quad factorialRecCaso0_\mbf(client) \\
\quad \qquad \lceil k == 0 \rceil \\
\quad factorialRecCasoN_\mbf(client, k, createNewMailbox()) \\
) \parallel \\
\quad  \rightarrow HACERALGOCON(\mbc) \\ 
)
\end{gather*}


\bibliography{references}{}
\bibliographystyle{plain}
\end{document}
