\begin{center}
\begin{LARGE}\textbf{Resumen}\end{LARGE}
\end{center}

\noindent
Las aplicaciones, con el incremento de la cantidad de núcleos por microprocesador, hacen un uso mas frecuente de la concurrencia. Una forma de atacar este tipo de problemas es el modelo tradicional de concurrencia que se basa en multi-hilos, variables compartidas, locks, etc. Este trabajo propone explorar un enfoque diferente: el modelo de actores, utilizado en la industria particularmente en lenguajes como Erlang y Scala con la librería Akka.

\noindent
El objetivo de este trabajo es comprender el modelo de actores y su semántica. Una buena herramienta para asistir a este proceso es utilizar métodos formales. Se propone modelar su semántica en \CSP y efectuar algunas pruebas utilizando la herramienta \FDR.

