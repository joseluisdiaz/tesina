\begin{center}
\begin{LARGE}\textbf{Resumen}\end{LARGE}
\end{center}

\noindent
Las aplicaciones, con el incremento de la cantidad de núcleos por microprocesador, hacen un uso más frecuente de la concurrencia. Una forma de atacar este tipo de problemas es el modelo tradicional de concurrencia que se basa en multi-hilos, variables compartidas, locks, etc. Este trabajo propone explorar un enfoque diferente: el modelo de actores utilizado en la industria, particularmente en lenguajes como Erlang y Scala con la librería Akka.

\noindent
El objetivo de este trabajo es comprender el modelo de actores y su semántica. Se exploran las características del modelo de actores, se introduce un lenguaje simple de actores llamado \SAL y se muestra su gramática.

\noindent
Se exponen algunos ejemplos en \SAL y su equivalente en \CSP, de esta manera se va construyendo un modelo que concluye con una semántica de \SAL en \CSP. Con este modelo se realizan algunas pruebas utilizando la herramienta \FDR, tales como revisión del árbol de ejecución y verificación de refinamiento utilizando el modelo de traza.



