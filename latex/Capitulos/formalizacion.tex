\chapter{Formaliznado la semantica en CSP}

TODO: Definir \textbf{translateExp}

La clase \textbf{Cmnd} con elementos de tipo S está dada por:

\begin{verbatim}
S :== S_1 ; S_2 | if b then S_1 else S_2 | send [e1, .., e_i] to a | become new
E(e_1, .. ,e_i) | let a_1 = new E_1(e_1,..,e_i) and ... a_j = new
E_1(e_1,..,e_i) { S } 
\end{verbatim}

definimos la funcion \textbf{translateCmd} de la siguiente forma:

\begin{verbatim}
translateCmd (S_1 S_2) = translateCmd(S_1) -> translateCmd(S_2)
\end{verbatim}


\begin{verbatim}
translateCmd(if b then S_1 else S_2) = 
   if (translateExp(b)) then
       translateCmd(S_1) else 
       translateCmd(S_2)
\end{verbatim}

\begin{verbatim}
translateCmd(send[e_1, ..., e_i] to a) = CommSend.a.
         (translateExp(e_1), ..., 
          translateExp(e_i)) 
\end{verbatim}

\begin{verbatim}
translateCmd(become new Beh(e_1, ..., e_n)) = runningBeh(self, e_1, ..., e_n)
\end{verbatim}

newEnv es el resultado de agregar a el entorno de las variables de mailbox $a_1
= E_1.pid_1$ .. $a_n = E_n.pid_n$
\begin{verbatim}
translateCmd(let a_1 = new E_1(e_1, ..., e_j) and 
         ... and a_n = new E_N(e_1, ..., e_j) { S } = 

CreateAsk?E_1.pid_1!(translateExp(e_1), ...,translateExp(e_j)) ->
CreateAsk?E_N.pid_n!(translateExp(e_1), ...,translateExp(e_j)) ->
translateCmd(S, newEnv)
\end{verbatim}


La clase \textbf{Beha} con elementos de tipo S está dada por:

\begin{verbatim}
def behName(a_1 .. a_i)[n_1 ... n_j]
  S
end def
\end{verbatim}

Tendria como equivalente en CSP:

\begin{verbatim}

behName = ||| actorId : {|BehName|} @ Create.actorId?(a_1, ..., a_i) ->
behNameRunning(actorId, a_1, .., a_n)

behNameRunning(self, a_1, .., a_n) = CommRecv.self(n_1 ... n_j) -> translateCmd(S)

\end{verbatim}
