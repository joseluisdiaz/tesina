\chapter{Preliminares}

\section{CSP}

\subsection{Paralelismo en CSP}

Para iniciar $n$ procesos de tipo $P$ escribimos los siguientes procesos en \CSP:

\begin{align*}
P =& \texttt{comportmiento-de-P} \then STOP \\
P_1 =& Iniciar_1 \then P \\
P_2 =& Iniciar_2 \then P \\
&\ldots \\
P_n =& Iniciar_n \then P \\
\end{align*}

Como vimos en el capitulo anterior, \CSP es sincrónico, mientras qué, el paso de mensajes o envío de comunicaciones en el sistema de actores no lo es. Si queremos transmitir entre dos procesos información, en \CSP lo escribimos de la siguiente forma:

\begin{align*}
P_1 = canal!1 \then& STOP \\
P_2 = canal?x \then& STOP \\
SYSTEM = P_1 \Parallel& P_2 \\
\end{align*}

Donde $P1$ le envía un uno a $P2$. Para poder desacoplar el envío de la recepción del mensaje, se puede utilizar una estructura intermedia de $BUFFER$, la escribimos de la siguiente forma:

\begin{align*}
BUFFER =& enviar?x \then recibir!x \then BUFFER_1 \\
P_1 =& enviar!1 \then STOP \\
P_2 =& recibir?x \then STOP \\
SYSTEM =& ( P_1 \Parallel P_2 ) \Parallel BUFFER \\
\end{align*}

Con estas estructuras, tendríamos los elementos básicos para poder crear un proceso y enviar una comunicación de manera asicronica. Se puede ver esto en el siguiente ejemplo:

\begin{align*}
BUFFER_1 =& enviar.1?x \then recibir.1!x \then BUFFER\\
BUFFER_2 =& enviar.2?x \then recibir.2!x \then BUFFER\\
SUMA =& inicia_{suma} \then recibir.1?x \then enviar.2?(x + 1) \then STOP \\
CLIENTE =& inicia_{suma}\then enviar.1!2 \then recibir.2?x \then STOP \\
BUFFER =& BUFFER_1 \Parallel BUFFER_2 \\
SYSTEM =& (SUMA \Parallel CLIENTE) \Parallel (BUFFER)\\t
\end{align*}

En el ejemplo anterior, $CLIENT$ inicia el proceso $SUMA$, y le envía un dos. Este envío es asíncrono por $BUFFER$. Cuando $SUMA$ recibe este dos, crea una nuevo mensaje y se lo envía a $CLIENTE$ de manera asincrónica, con el valor que recibió incrementado en uno.

Este simple ejemplo expone algunos de los problemas que se resolverán en este capitulo, tales como: enviar una comunicación a un proceso especifico, crear varios tipos de procesos.

\section{CSPm}

