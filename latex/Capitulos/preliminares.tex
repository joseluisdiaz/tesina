\chapter{Preliminares}

En en la primera sección de este capítulo se exploran algunas particularidades del paralelismo en \CSP, tales como, paralelismo sincrónico, alfabetizado, entrelazado y generalizado. En la segunda sección se muestran algunas construcciones en \CSPm que resultan útiles. 

La sección de \CSP no pretende ser una introducción al lenguaje; se asume que el lector tiene cierta familiaridad con él. Para una introducción se puede consultar \cite{Cristia:CSP}, para una referencia completa \cite{Roscoe:1997:TPC:550448}

\section{Algunas construcciones de CSP}

En esta sección se muestran algunos de los operadores de paralelismo de \CSP, seguido de algunos ejemplos de uso y su relación con el modelo de actores.  

\subsubsection*{Paralelismo sincrónico}

El operador más simple de \CSP es el que está dispuesto a sincronizar todos los eventos. Es decir, ambos procesos compuestos por este operador avanzan cuando encuentran un evento que ambos están dispuestos a sincronizar. Por ejemplo:

\begin{align*}
P_1 =& a \then P_1\\
P_2 =& a \then P_2 \\
SYSTEM =& P_1 \parallel P_2
\end{align*}
donde $P_1$ y $P_2$ sincronizan en el evento $a$. 

Cuando utilizamos procesos parametrizados muchas veces es útil enviar información. El siguiente ejemplo muestra esto:

\begin{align*}
P_1 =& canal!1 \then STOP \\
P_2 =& canal?x \then Q(x) \\
SYSTEM =& P_1 \parallel P_2 \\
\end{align*}
donde $canal!1$ es quien envía el valor $1$,  $canal?x$ lo recibe y $Q$ es un proceso parametrizado. Para entender un poco más cómo funciona la notación que involucra $\langle ? \rangle$ y $\langle ! \rangle$, supongamos que $x$ puede tomar los valores $1$, $2$ y $3$. La expresión $canal?x$ equivale a un proceso que está dispuesto a sincronizar con todos estos potenciales valores:
\begin{align*}
P_2 & =  canal.1 \then STOP \\
      & \Extchoice canal.2 \then STOP \\
      & \Extchoice canal.3 \then STOP 
\end{align*}

Como $x$ es una variable libre y el evento que termina sincronizando es $canal.1$, ésta toma el valor $1$. En realidad el paso de información es ficticio; todo el tiempo se está sincronizando en eventos.

\subsubsection*{Paralelismo alfabetizado}

Mientras más procesos combinemos utilizando el operador $\parallel$, más procesos tienen que ponerse de acuerdo en los eventos a sincronizar si se pone en paralelo los procesos $P$ y $Q$ no necesariamente todas las comunicaciones de $P$ son para $Q$.

Si $X$ e $Y$ son dos conjunto de eventos, $P\ \textsubscript{X}\parallel\textsubscript{Y}\ Q$ es la composición paralela en donde $P$ tiene sólo permitido comunicar los eventos $X$ y donde $Q$ tiene sólo permitido comunicar los eventos $Y$, y únicamente tienen que ponerse de acuerdo en la intersección $X \cap Y$. Por ejemplo:

\[ 
 ( a \then b \then b \then STOP )\  \textsubscript{\{a, b\}} \parallel \textsubscript{\{b, c\}}\ ( b \then c \then b \then STOP ) 
\]

Se comporta como:

\[ 
 ( a \then b \then c \then b \then STOP )
\]

\subsubsection*{Entrelazado}
Los operadores $\parallel$ y $\textsubscript{X}\parallel\textsubscript{Y}$ tienen la propiedad que todos los procesos involucrados tienen que sincronizar algún evento. Utilizando el operador de entrelazado ($\Interleave$), cada proceso corre independiente de cualquier otro. Se nota $P \Interleave Q$.

\subsubsection*{Paralelismo generalizado}
Existe una forma general de escribir todos los operadores vistos utilizando el operador de paralelismo generalizado $P \Parallel\limits_{X} Q$, donde $P$ y $Q$ solo tienen que ponerse de acuerdo en los eventos pertenecientes a $X$ y los eventos que están por fuera de $X$ se procesan independientemente.

Podemos escribir el operador de entrelazado usando la siguiente equivalencia:

\[
 P \Interleave Q = P \Parallel\limits_{\{\}} Q
\]

Podemos escribir el operador de paralelismo alfabetizado como:

\[
 P\ \textsubscript{X}\parallel\textsubscript{Y}\ Q = P \Parallel\limits_{X \cap Y} Q
\]

Se puede definir el operador de paralelismo sincrónico de la siguiente forma, donde $\Sigma$ son todos los eventos posibles en un sistema dado.
\[
 P \parallel Q = P \Parallel\limits_{\Sigma} Q
\]

\subsubsection*{Cadena de caracteres}
\CSP no tiene soporte para cadena de caracteres, en esencia solo son procesos y eventos. Será útil en el capítulo que se define el modelo poder utilizar dentro de las secuencias algún tipo de cadena de carácteres. Estas son inmutables y la única operación que se define sobre ellas es la comparación. Se representan utilizando una secuencia alfanumérica, encerradas entre comillas simples. Por ejemplo:
\begin{gather*}
'cadena1' \\
'cadena2' \\
\langle 'cadena1', 'cadena2' \rangle
\end{gather*}

\subsubsection*{Actores y CSP}\label{preliminares:actores}

Como vimos en el capítulo anterior, \CSP es sincrónico, mientras que el paso de mensajes o envío de comunicaciones en el sistema de actores no lo es. Si se quiere transmitir información entre dos procesos en \CSP lo escribimos (como vimos en el apartado de paralelismo sincrónico),de la siguiente forma:
\begin{align*}
P_1 =& canal!1 \then STOP \\
P_2 =& canal?x \then STOP \\
SYSTEM =& P_1 \Parallel P_2  
\end{align*}

Para poder desacoplar el envío de la recepción del mensaje, se puede utilizar una estructura intermedia de $BUFFER$, cuya especificación es:
\begin{align*}
BUFFER =& enviar?x \then recibir!x \then BUFFER \\
P_1 =& enviar!1 \then STOP \\
P_2 =& recibir?x \then STOP \\
SYSTEM =& ( P_1 \Interleave P_2 ) \Parallel BUFFER \\
\end{align*}

Como la comunicación es desde $P_1$ hacia $BUFFER$ y desde $BUFFER$ hacia $P_2$, no hay ninguna comunicación entre $P_1$ y $P_2$. Por esto se utiliza el operador de entrelazado.

En \CSP no existe el concepto de instancia, y se debe definir la red de procesos desde el comienzo. Es decir, no podemos crear nuevos procesos desde un proceso. Por lo tanto, simularemos la creación activando cada instancia mediante un evento especial. Para iniciar $n$ procesos de tipo $P$ se escriben los siguientes procesos en \CSP:
\begin{align*}
P =& \texttt{comportamiento-de-P} \then STOP \\
P_1 =& iniciar_1 \then P \\
P_2 =& iniciar_2 \then P \\
&\ldots \\
P_n =& iniciar_n \then P \\
SYSTEM =& P_1 \Interleave P_2 \Interleave \ldots \Interleave P_n
\end{align*}
donde $iniciar_i$ es el evento especial que da inicio a $P$. Se utiliza el operador de entrelazado para combinar los procesos $P_1, P_2, \ldots, P_n$, ya que no tiene que haber ninguna comunicación entre ellos. En el caso de existir algún evento compartido, este debería ser desestimado y no generar un encuentro. 
Con estos elementos podemos crear un proceso y enviar una comunicación de manera asincrónica. Se puede ver esto en el siguiente ejemplo:

\begin{align}\label{eq:smallactor}
\begin{split}
BUFFER_1 &= enviar.1?x \then recibir.1!x \then BUFFER_1\\
BUFFER_2 &= enviar.2?x \then recibir.2!x \then BUFFER_2\\
INC &= inicia_{inc} \then recibir.1?x \then enviar.2!(x + 1) \then STOP \\
CLIENTE &= inicia_{inc}\then enviar.1!2 \then recibir.2?x \then STOP \\
BUFFER &= BUFFER_1 \Interleave BUFFER_2 \\
SYSTEM &= (INC \Parallel\limits_{\{inicia_{inc}\}} CLIENTE) \Parallel\limits_{Y}\ BUFFER
\end{split}
\end{align}

Es decir que $CLIENTE$ inicia el proceso $SUMA$, y le envía un $2$. Este envío es asíncrono por $BUFFER_1$. Cuando $SUMA$ recibe este $2$, crea una nuevo mensaje y se lo envía a $CLIENTE$ de manera asincrónica, con el valor que recibió incrementado en uno. En la composición de $SYSTEM$ se puede ver que el único evento que se sincroniza entre $SUMA$ y $CLIENTE$ es $inicia_{suma}$. En el otro operador paralelo, los valores de $Y$ vienen dados por los eventos en los que la composición de $SUMA$ y $CLIENT$ sincronizan con $BUFFER$. Para esto deberíamos saber qué valores puede tomar $x$. Asumiendo que toma los valores $1$, $2$ y $3$. Los eventos a sincronizar serían el conjunto generado por $\{m: [1,2] ,n: [1,2,3] | recibir.m.n \} \cup \{m: [1,2] ,n: [1,2,3] | enviar.m.n \} $, es decir todos los eventos inherentes a $BUFFER$.

Este último ejemplo muestra dos de los aspectos que se desarrollarán en el capítulo siguiente: cómo desacoplar el envío de mensajes y cómo simular la creación de un proceso. También puede verse el uso de los distintos operadores paralelo.
 
\section{El lenguaje CSPm}

\CSPm es un lenguaje funcional, que tiene una integración para definir procesos de \CSP. También permite realizar aserciones sobre los procesos de \CSP resultantes. Este lenguaje es el que utiliza la plataforma \FDR. En esta sección se describirán algunas de las construcciones de \CSP en \CSPm y algunas construcciones propias de \CSPm.

\subsubsection{Tipos algebraicos}

Permite declarar tipos estructurados, son similares a las declaraciones de tipo \textit{data} de Haskell. La más simple de las declaraciones es utilizando constantes.

\begin{verbatim}
datatype ColorSimple = Rojo | Verde | Azul
\end{verbatim}

Esto declara \verb=Rojo=, \verb=Verde= y \verb=Azul=, como símbolos del tipo \verb=ColorSimple=, y vincula \verb=ColorSimple= al conjunto \verb={Rojo, Verde, Azul}=. Estos tipos de datos puede tener parámetros. Por ejemplo, se pude agregar un constructor de datos \verb=RGB=, a saber:

\begin{verbatim}
datatype ColorComplejo = Nombre.ColorSimple | RGB.{0..255}.{0..255}.{0..255}
\end{verbatim}

Esto declara \verb=Nombre=, como un constructor de datos, de tipo \verb=ColorSimple= $\Rightarrow$ \verb$ColorComplejo$ y a \verb$RGB$ como un constructor de datos de tipo \verb$Int$ $\Rightarrow$ \verb$Int$ $\Rightarrow$ \verb$Int$ $\Rightarrow$ \verb$ColorComplejo$ y \verb$ColorComplejo$ es el conjunto:
\begin{multline*}
\{ Nombre.c | c \leftarrow  ColorSimple \} \cup \{ RGB.r.g.b |  r \leftarrow \{ 0 \dots 255 \},  g \leftarrow \{ 0 \dots 255 \}, \\  b \leftarrow \{ 0 \dots 255 \} \}  
\end{multline*}

Si se declara un tipo de datos T, entonces a T se adjunta el conjunto de todos los valores de tipo de datos posibles que se pueden construir. 

\subsubsection{Canales}

Los canales de \CSPm son utilizados para crear eventos, y se declaran de una manera similar a los tipos de datos. Por ejemplo:

\begin{verbatim}
channel estaListo
channel x, y : {0..1}.Bool
\end{verbatim}

Declara tres canales, uno que no toma parámetros (\verb=estáListo= es de tipo Event), y dos que tienen dos componentes. Cualquier valor del conjunto \verb={0,1}= y un booleano. El conjunto de los eventos definidos es el siguiente: 
\begin{verbatim}
{ estaListo, x.0.false, x.1.false, x.0.true, x.1.true, y.0.false, 
y.1.false, y.0.true, y.1.true }	
\end{verbatim}
Estos eventos pueden ser parte de la declaración de procesos como por ejemplo:
\begin{verbatim}
{P = x?a?b -> STOP}
\end{verbatim}

\subsubsection{Búsqueda de patrones}

Es posible en \CSP que los valores puedan ser buscados por coincidencia de patrones. Por ejemplo, la siguiente función toma un entero, se puede usar en la búsqueda de patrones para especificar un comportamiento diferente dependiendo de este argumento:

\begin{verbatim}
f(0) = True
f(1) = False
f(_) = error("Error")
\end{verbatim}

Funciona de manera similar a Haskell, también se pueden utilizar otras construcciones como secuencias o algún tipo algebraico.

\subsubsection{Operadores replicados}
\FDR tiene una versión replicada o indexada de alguno de sus operadores. Estos proveen una forma simple de construir un proceso que consiste en una serie de procesos compuestos utilizando el mismo operador. Por ejemplo se define P de la siguiente manera, \verb=P :: (Int) -> Proc= luego, \verb=||| x: { 0..2 } @ P(x)= evalúa P para cada valor de x en el conjunto dado y los compone utilizando el operador de entrelazado. Por lo tanto, lo anterior es equivalente a \verb=P (0) ||| P (1) ||| P (2)=.


La forma general de un operador replicado es:

\begin{verbatim}
op <declaraciones> @ P
\end{verbatim}

donde op es un operador que puede ser el de entrelazado, selección interna, etc. P es la definición de un proceso, puede hacer uso de las variables definidas por las declaraciones. Cada uno de los operadores evalúa P para cada valor que toman las declaraciones antes de componerlas juntas usando op.

En \CSPm las declaraciones pueden ser construcciones por comprensión, cómo las que generan nuevos conjuntos basados en conjuntos existentes. Por ejemplo, el conjunto por comprensión: \verb={x + 1 | x <- xs}= incrementa cada elemento del conjunto xs en 1. 

Otra forma de construcción por compresión es utilizar la siguiente expresión: \verb={| CANAL |}=, que genera el conjunto de todos los elementos del canal. Por ejemplo, dada la siguiente definición de canal \verb=channel X : {0..1}.Bool=, el conjunto generado por \verb={| X |}=es: \verb={X.0.false, X.0.true, X.1.false, X.1.true}=.



\subsubsection{Otras construcciones}
En esta sección se muestran algunas de las conversiones útiles para entender un programa en \CSPm.

Tabla de conversión para secuencias:

\begin{center}
\begin{tabular}{ l l }
  $\seq$ a & \verb|Seq(a)| \\
  $\nil$ & \verb|< >| \\
  $\langle 1, 2, 3 \rangle$ & \verb|<1,2,3>| \\
  $\Nil s$ & \verb=Null(s)=
  \\
  $s \cat t$ & \verb=s^t=
  \\
  $\# s$ & \verb|#s|
  \\
  $head~s$ & \verb|head(s)|
  \\
  $tail~s$ & \verb|tail(s)|
  \\
  $\dcat s$ & \verb|concat(s)|
  \\
  $x \elem s$ & \verb=elem(x,s)=
  \\
  $\ran s$ &  \verb|set(s)|
  
\end{tabular}
\end{center}

Tabla de conversión para la definición de procesos:

\begin{center}
\begin{tabular}{ l l }
  Stop & \verb=STOP= \\
  Skip & \verb=SKIP= \\
  $c \then p$ & \verb=c -> p= \\
  $c ? x  \then p$  & \verb=c?x -> p= \\
  $c ! v \then p$ & \verb=c!v -> p= \\
  $p \extchoice q$ & \verb=p [] q= \\
  $p \intchoice q$ & \verb=p |~| q= \\
  $p \interleave q $ & \verb=p ||| q= \\
  $p \parallel q $ & \verb=p || q= \\
  $p\ \textsubscript{X} \parallel \textsubscript{Y}\ q$ & \verb=p |[|X|]| q= \\
  $p \Parallel\limits_{X} q$ & \verb=p |[x||y]| q= \\
  \end{tabular}
\end{center}

 	