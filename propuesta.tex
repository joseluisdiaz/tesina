\documentclass{article}
\usepackage[utf8]{inputenc}
\usepackage{hyperref}
\usepackage{cite}

\begin{document}

\title{Propuesta de tesina para la obtención del grado
Licenciado en Ciencias de la Computación}

\maketitle

\section{Situación del postulante}
El postulante aprobó 29 materias de la Licenciatura en Ciencias de la Computación (plan 1994) en marzo de 2015. Solamente le resta la tesina.

\section{Título}
Un modelo semantico basado en CSP para Actores.

\section{Motivación y objetivo general}

Las aplicaciones, con el incremento de la cantidad de núcleos por microprocesador hacen un uso mas frecuente de la concurrencia. Una forma de atacar este tipo de problemas, es utilizando multi-hilos, variables compartidas, locks, etc. Este trabajo propone utilizar un enfoque diferente: el modelo de actores, utilizado en la industria particularmente en lenguajes como Erlang\cite{Cesarini:2009:EP:1717841} y Scala\cite{scala-overview-tech-report} con la librería Akka\cite{Wyatt:2013:AC:2663429}. 
Se intentará explicar muy brevemente la diferencia entre ambos modelos usando el problema del Jardin Ornamental. El enunciado del problema es el siguiente: supongamos que tenemos dos entradas a un parque, y se requiere saber cuanta gente ingresa, para eso se instalan un molinete en cada entrada. Se utiliza una computadora para registrar la información de ingreso.
Una implementación en un lenguaje imperativo con threads, incluiria una variable global que guarde cantidad de visitantes, dos threads representando los molinetes que incrementan esta variable. Sin ningún tipo de protección en la region critica planteada por la actualización de la variable se perderian incrementos, ya que cada thread carga localmente el valor de la variable global, efectua un incremento y finalmente guarda el valor en la variable global. 
Utilizando el modelo de actores, que tiene como único mecanismo de comunicacion entre procesos el paso de mensajes, este mismo problema puede ser representado utilizando un actor que realize la tarea de contador, incrementara su valor cuando reciba un mensaje \emph{inc}, y otros dos actores que emitirán estos mensajes (los molinetes). En este caso el problema de la perdida de la actualización no ocurre, por las garantias que tiene el paso de mensajes entre actores. Mas adelante regresaremos con esta idea. 
El objetivo de este trabajo es comprender en profundidad el modelo de actores y su semantica. Una buena herramienta para asistir a este proceso es utilizar metodos formales. Se propone modelar su semantica en CSP y efectuar algunas pruebas utilizando la herramienta FDR\cite{fdr}.


\section{Fundamentos y estado de conocimiento sobre el tema}
 
Como señala Rob Pike en su charla titulada ``Concurrencia no es paralelismo'' \cite{rpike13_cnp}, existe una diferencia conceptual entre estos dos, la cual muchas veces se pasa por alto. En programación, la concurrencia es la composición de los procesos independientemente de ejecución, mientras que el paralelismo es la ejecución simultánea de cálculos (posiblemente relacionados). 

El modelo de actores es originalmente propuesto por C. Heweeit\cite{Wyatt:2013:AC:2663429}, es un enfoque diferente a como estructurar programas concurrentes. Un actor, computacionalmente, es una entidad que de manera concurrente puede hacer:

\begin{itemize}
\item Enviar y recibir un numero finito de mensajes a otros actores
\item Crear un numero finito de actores
\item Designar un nuevo comportamiento a ser usado cuando se reciba el proximo mensaje.
\end{itemize}

Gul Agha\cite{Agha:1986:AMC:7929} en parte de su trabajo doctoral describe un modelo denotacional. Los actores usan una semantica operacional estructuada\cite{Plotkin81astructural}. Define dos tipos de transiciones que representan la evolución de la configuración de un sistema de actores, la primera \emph{transición posible} representa cuales son todas las posibles transiciones del sietema, como esta relación es insuficiente para garantizar la entrega de mensajes, tambien define \emph{transición siguiente} que expresa justamente esta garantía.

\emph{Communicating Sequential Processes} (CSP), fue propuesto por primera vez por C.A.R Hoare\cite{Hoare:1978:CSP:359576.359585}, es un lengueje para la especificación y verificacion del comportamiento concurrentes de sistemas. Como su nombre indica, CSP permite la descripción de sistemas en términos de componentes que operan de forma independiente que interactúan entre sí únicamente a través del paso paso de mensajes. Las relaciones entre los diferentes procesos y la forma en que cada proceso se comunica con su entorno, se describen utilizando un algebra de procesos.

Comparando $CSP$ con el modelo de actores, ambos mecanismos tienen procesos concurrentes que intercambian mensajes. Sin embargo, los dos modelos hacen algunas decisiones fundamentalmente diferentes con respecto a las primitivas que proporcionan:

\begin{itemize}
\item Los procesos de CSP son anónimos, mientras que los actores tienen identidades.
\item Los mensajes fundamentalmente consisten en un encuentro entre los procesos involucrados en el envío y la recepción del mensaje, es decir, el remitente no puede transmitir un mensaje hasta que el receptor está dispuesto a aceptarlo. Por el contrario, en los sistemas de actores, el paso de mensajes es fundamentalmente asíncrona, es decir, la transmisión y la recepción de mensajes no tienen que suceder al mismo instante.
\item CSP utiliza canales explícitos para el paso de mensajes, mientras que los sistemas de actores transmiten mensajes a los actores de destino.
\end{itemize}

Estos enfoques pueden ser considerados duales de uno al otro, en el sentido de que los sistemas basados en \emph{encuentros} pueden utilizarse para construir comunicaciones que se comporten como sistemas de mensajería asíncrona, mientras que los sistemas asíncronos se pueden utilizar para construir las comunicaciones sincronzas utilizando algún protocolo que permita el encuentro entre los procesos. Lo mismo ocurre con los canales.

El $\pi-calculus$, es un álgebra de procesos que captura en esencia la simpleza del $\lambda-calculo$, un posterior trabajo de Gul Agha\cite{apicalculus} crea un sustento teórico algebraico para el modelo de actores, y le da semantica a un lenguaje concreto llamado SAL (Simple Actor Language).

FDR es una herramienta para el análisis de los programas escritos en notación CSP de Hoare, en particular utilizando $CSP_M$, que combina los operadores de $CSP$ con un lenguaje de programación funcional. FDR original fue escrito en 1991 por Formal Systems (Europe) Ltd, que también lanzo la version 2 a mediados de la década de 1990. La versión actual de la herramienta es FDR3, se liberó por primera vez en 2013 por la Universidad de Oxford, que también liberó versiones FDR2 2.90 y superiores en el período 2008-12.

\section{Objetivos específicos}

\begin{itemize}

  \item Realizar un estudio exhaustivo del modelo de actores. 
  \item Crear una semantica del modelo de actores en CSP, adaptar la misma al lenguaje SAL.
  \item Modelar en SAL, algunos programas tradicionales como el factorial, productor consumidor con buffer acotado, una pila.
  \item Generar el modelo asociado en CSP, y hacer pruebas utilizando la herramienta FDR3.

\end{itemize}


\section{Metodología y plan de trabajo}

Los objetivos específicos en sí, describen una secuencia adecuada del trabajo a realizar el cual parecería tener una carga equilibrada en complejidad y en tiempo de ejecución, se propone trabajar durante 6 meses con una dedicación de 20 horas semanales. Se finalizará con la escritura de lo encontrado.

\bibliography{references}{}
\bibliographystyle{plain}
\end{document}
